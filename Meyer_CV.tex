% !TEX TS-program = XeLaTeX
\documentclass[11pt]{res}

\begin{document}

%----------------------------------------------------------------------------------------
%	YOUR NAME AND ADDRESS(ES) SECTION
%----------------------------------------------------------------------------------------

\thispagestyle{firststyle}

\name{Aaron S. Meyer\\ \\} % Your name at the top

\address{aameyer@mit.edu \\ (661) 978-2606 } % Your address 1

\address{77 Massachusetts Avenue, 76-361F
 \\ Cambridge, MA 02139} % Your address 2

%----------------------------------------------------------------------------------------

\begin{resume}

\raggedright 

%	EDUCATION SECTION
\section{Education}

{\sl Ph.D.}, 
Biological Engineering  \hfill April~2014\\ 
Massachusetts Institute of Technology, Cambridge, MA \\ 
Thesis: Quantitative approaches to understanding signaling regulation of 3D cell migration
 
{\sl B.S.}, Bioengineering, magna cum laude \hfill June~2009\\ 
University of California, Los Angeles, CA

%%%%%%%%%%%%%%%%
\section{Research Experience}

{\sl Principal Investigator \& Research Fellow} \hfill September 2014 -- Present \\
Koch Cancer Institute, MIT, Cambridge, MA 
\begin{itemize} \itemsep -2pt % Reduce space between items
\item Developing systems cancer cell resistance model allowing prognostic therapeutic design
\item Refining models of TAM receptor activation for optimal immunotherapeutic targeting
\item Developing multiplexed methods for protein-protein interaction analysis
\end{itemize}

{\sl Postdoctoral Associate in the labs of Forest White \& Douglas Lauffenburger} \hfill June -- September 2014 \\
Department of Biological Engineering \& Koch Cancer Institute, MIT, Cambridge, MA 
\begin{itemize} \itemsep -2pt % Reduce space between items
\item Performed global phosphotyrosine analysis of TAM receptor transactivation 
\item Utilized systems analysis to identify receptor localization importance to transactivation effects
\end{itemize}
 
{\sl Graduate Researcher in the labs of Douglas Lauffenburger \& Frank Gertler} \hfill 2009 -- 2014 \\
Department of Biological Engineering \& Koch Cancer Institute, MIT, Cambridge, MA 
\begin{itemize} \itemsep -2pt % Reduce space between items
\item Identified similarities in migration response between dimensionalities, suggesting relevant migration  assays for invasive disease
\item Studied transactivation of TAM receptors and its role in promoting motility response
\item Developed systems models of TAM signaling, unifying conflicting observations regarding the receptors in normal biology and suggesting new methods of intervention to modulate activity
\end{itemize}

{\sl Undergraduate Researcher in the lab of Daniel Kamei} \hfill 2006 -- 2009 \\ 
Department of Bioengineering, University of California, Los Angeles, CA 
\begin{itemize} \itemsep -2pt % Reduce space between items
\item Investigated biomarker purification using novel aqueous micellar systems
\item Extended a previous statistical mechanics model to nucleic acid partitioning
\item Designed and executed experiments to analyze the partitioning of surfactant systems
\item Developed assays for quantifying the concentration of charged and uncharged surfactants
\end{itemize}

{\sl Summer Intern, Bioprocess Development Division} \hfill 2008 \\ 
Schering-Plough Corporation, Watchung, NJ
\begin{itemize} \itemsep -2pt % Reduce space between items
\item Developed a novel method for high-throughput batch culture within deep-welled microtiter plates
\item Investigated the social behavior of nonproducing impurities within monoclonal cultures
\item Provided statistical basis for process-based confidence in monoclonality
\end{itemize}





% PUBLICATIONS SECTION
\section{Refereed Publications}
{ \leftskip 0.1in
\parindent -0.1in


Manole, S., {\bf A.S. Meyer}. ``Conserved RTK-intrinsic signaling consequences result in distinct bypass resistance capacity dependent upon pathway dependencies." {\sl Submitted}.

\parskip 0.1in

McConnell, R.E., J.E. Van Veen, M. Vidaki, A.V. Kwiatkowski, {\bf A.S. Meyer}, D.A. Lauffenburger, F.B. Gertler. ``Filopodia elongation towards the chemorepellent SLIT is required for axon repulsion." {\sl Submitted}.

Miller, M.A., M.J. Oudin, R.J. Sullivan, D.T. Frederick, {\bf A.S. Meyer}, S. Wang, H. Im, J. Tadros, L.G. Griffith, H. Lee, R. Weissleder, K.T. Flaherty, F.B. Gertler, D.A. Lauffenburger. ``Reduced proteolytic shedding of receptor tyrosine kinases is a post-translational mechanism of kinase inhibitor resistance." {\sl Submitted}.

Miller, M.A., M. Moss, G. Powell, R. Petrovich, L. Edwards, {\bf A.S. Meyer}, L.G. Griffith, D.A. Lauffenburger. ``Targeting autocrine HB-EGF signaling with specific ADAM12 inhibition using recombinant ADAM12 prodomain." {\sl Accepted. Scientific Reports} (2015).

{\bf Meyer\footnote{Co-corresponding authors.}, A.S.}, A.J.M. Zweemer, D.A. Lauffenburger\footnotemark[\value{footnote}]. ``The AXL receptor is a sensor of ligand spatial heterogeneity." {\sl Cell Systems} (2015).

Riquelme, D.N., {\bf A.S. Meyer}, M. Barzik, A. Keating, F.B. Gertler. ``Selectivity in subunit composition of Ena/VASP tetramers." {\sl Bioscience Reports} (2015).

{\bf Meyer, A.S.}, M.A. Miller, F.B. Gertler, D.A. Lauffenburger. ``The receptor AXL diversifies EGFR signaling and limits the response to EGFR-targeted inhibitors in triple-negative breast cancer cells." {\sl Science Signaling} (2013).

Miller\footnote{Equally contributing authors.}, M.A., {\bf A.S. Meyer}\footnotemark[\value{footnote}], M. Beste, Z. Lasisi, S. Reddy, K. Jeng, C.-H. Chen, J. Han, K. Isaacson, L.G. Griffith, D.A. Lauffenburger. ``ADAM-10 and -17 regulate endometriotic cell migration via concerted ligand and receptor shedding feedback on kinase signaling." {\sl Proc. Natl. Acad. Sci. U.S.A.} (2013).

{\bf Meyer, A.S.}, S.K. Hughes-Alford, J.E. Kay, A. Castillo, A. Wells, F.B. Gertler, D.A. Lauffenburger. ``2D protrusion but not motility predicts growth factor-induced cancer cell migration in 3D collagen." {\sl J. Cell Biol.} (2012).

Kim, H.D., {\bf A.S. Meyer}, J.P. Wagner, S.K. Alford, A. Wells, F.B. Gertler, D.A. Lauffenburger. ``Signaling network state predicts Twist-mediated effects on breast cell migration across diverse growth factor contexts." {\sl Mol. Cell. Proteomics} (2011).

{\bf Meyer, A.S.}, R.G. Condon, G. Keil, Jhaveri, N., Liu, Z., Tsao, Y.-S., \href{http://www.ncbi.nlm.nih.gov/pubmed/21954223}{``Fluorinert, an oxygen carrier, improves cell culture performance in deep square 96-well plates by facilitating oxygen transfer."} {\sl Biotechnol. Prog.} (2012).

Mashayekhi, F., {\bf A.S. Meyer}, S.A. Shiigi, V. Nguyen and D.T. Kamei, \href{http://www.ncbi.nlm.nih.gov/pubmed/19061237}{``Concentration of mammalian genomic DNA using two-phase aqueous micellar systems."} {\sl Biotechnol. Bioeng.} (2009).

}

%----------------------------------------------------------------------------------------


\section{Research Support \& Awards}

{\sl Frontier Research Program Initiator Award} \hfill 2015 \\
Koch Institute for Integrative Cancer Research\\
``Multiplexed Tools for Probing Chemokine Receptor Activation State in Breast Cancer"

{\sl NIH Director's Early Independence Award}  \hfill 2014--2019 \\
DP5-OD019815 -- ``Adapter-Layer RTK Signaling: Basic Understanding \& Targeted Drug Resistance"

{\sl Siebel Scholar, Class of 2014} \hfill 2013 

{\sl Whitaker Fellowship}  \hfill 2013\\
Massachusetts Institute of Technology

{\sl Repligen Fellowship in Cancer Research} \hfill 2012 \\
Koch Institute for Integrative Cancer Research 

{\sl Frontier Research Program Initiator Award} \hfill 2011 \\
Koch Institute for Integrative Cancer Research\\
``Global Growth Factor Reprogramming and Invasion By AXL Expression And Shedding In Breast Carcinoma"

{\sl Breast Cancer Research Predoctoral Fellowship} \hfill 2010 -- 2014 \\
Department of Defense\\
W81XWH-11-1-0088 -- ``Molecular Regulatory Network Dysregulation in Breast Cancer Cell Migration \& Invasion"

{\sl Graduate Research Fellowship}  \hfill 2009 -- 2014 \\
National Science Foundation

{\sl Momenta Presidential Fellowship} \hfill 2009 \\
Massachusetts Institute of Technology



\section{Teaching \& Mentoring Experience}

{\sl Faculty of the Citizen Science Program} \hfill July 2015 -- January 2016 \\
Bard College, Citizen Science Program, Annandale-on-Hudson, NY	
\begin{itemize} \itemsep -2pt % Reduce space between items
\item Leading a short course introducing students to the natural sciences and scientific method
\end{itemize}

{\sl Undergraduate Mentor} \hfill 2009 -- Present \\
MIT, Department of Biological Engineering, Cambridge, MA
\begin{itemize} \itemsep -2pt % Reduce space between items
\item Designed and supervised projects for nine undergraduate students
\end{itemize}
	

{\sl Teaching Assistant}, Thermodynamics of Biomolecular Systems \hfill 2010 \\
MIT, Department of Biological Engineering, Cambridge, MA
\begin{itemize} \itemsep -2pt % Reduce space between items
\item Taught at weekly discussion sections, office hours, and individual appointments
\item Helped write and graded problem sets and exam questions
\end{itemize}




\section{Conference \& Invited Presentations}

{\sl Systems Approaches to Cancer Biology}, NCI Invited Oral Presentation \hfill April 2016 \\
Manole,~S., {\bf A.S.~Meyer} ``Conserved RTK-intrinsic signaling results in distinct bypass resistance capacity and suggests therapeutic opportunities."

{\sl Biomedical Engineering Society Annual Meeting} \hfill October 2015 \\
Manole, S., {\bf A.S. Meyer}. ``Conserved RTK-intrinsic signaling consequences result in distinct bypass resistance capacity dependent upon pathway dependencies."

{\sl ICBP Principal Investigators Meeting} \hfill May 2015 \\
Manole, S., {\bf A.S. Meyer}. ``Conserved RTK-intrinsic signaling consequences result in distinct bypass resistance capacity dependent upon pathway dependencies."

\clearpage

{\sl NIH Common Fund High-Risk High-Reward Symposium} \hfill December 2014 \\
{\bf A.S. Meyer}. ``Adapter-Layer Integration of RTK Signaling: Basic Understanding and Application to Prediction of Targeted Drug Resistance."

{\sl Biomedical Engineering Society Annual Meeting}, Selected Oral Presentation \hfill October 2014 \\
{\bf Meyer, A.S.}, C.A. Riley, D.A. Lauffenburger. ``AXL Is a Spatial Ligand Differentiation Sensor."

{\sl Interdisciplinary Signaling Workshop}, Selected Oral Presentation \hfill July 2014 \\
{\bf Meyer, A.S.}, C.A. Riley, D.A. Lauffenburger. ``AXL Is a Spatial Ligand Differentiation Sensor."

{\sl ICBP Principal Investigators Meeting} \hfill May 2014 \\
{\bf Meyer, A.S.}, C.A. Riley, D.A. Lauffenburger. ``AXL is a spatial ligand differentiation sensor."

{\sl AACR Molecular Targets and Cancer Therapeutics} \hfill October 2013 \\
{\bf Meyer, A.S.}, F.B. Gertler, D.A. Lauffenburger. ``AXL amplifies EGFR signaling and drives resistance in triple negative breast carcinoma cells."

{\sl Merrimack Pharmaceuticals}, Invited Oral Presentation \hfill October 2013 \\
{\bf Meyer, A.S.}, C.A. Riley, D.A. Lauffenburger. ``AXL is a spatial ligand differentiation sensor."

{\sl ICBP Principal Investigators Meeting} \hfill May 2013 \\
{\bf Meyer, A.S.}, F.B. Gertler, D.A. Lauffenburger. ``AXL amplifies EGFR signaling and drives resistance in triple negative breast carcinoma cells."

{\sl Merrimack Pharmaceuticals}, Invited Oral Presentation \hfill January 2013 \\
{\bf Meyer, A.S.}, F.B. Gertler, D.A. Lauffenburger. ``AXL amplifies EGFR signaling and drives resistance in triple negative breast carcinoma cells."

{\sl PTMs in Cell Signaling, Copenhagen Bioscience Conferences}, Travel Award \hfill December 2012 \\
{\bf Meyer, A.S.}, F.B. Gertler, D.A. Lauffenburger. ``AXL amplifies EGFR signaling and drives resistance in triple negative breast carcinoma cells."

{\sl Biomedical Engineering Society Annual Meeting}, Selected Oral Presentation \hfill October 2012 \\
{\bf Meyer, A.S.}, S.K. Hughes-Alford, J.E. Kay, A. Castillo, A. Wells, F.B. Gertler, D.A. Lauffenburger. ``2D protrusion but not motility predicts growth factor-induced cancer cell migration in 3D collagen."

{\sl Signaling of Adhesion Receptors, Gordon Research Conference} \hfill June 2012 \\
{\bf Meyer, A.S.}, S.K. Hughes-Alford, J.E. Kay, A. Castillo, A. Wells, F.B. Gertler, D.A. Lauffenburger. ``2D protrusion but not motility predicts growth factor-induced cancer cell migration in 3D collagen."

{\sl Systems Biology of Human Disease}, Travel Award \hfill May 2012 \\
{\bf Meyer, A.S.}, S.K. Hughes-Alford, J.E. Kay, A. Castillo, A. Wells, F.B. Gertler, D.A. Lauffenburger. ``2D protrusion but not motility predicts growth factor-induced cancer cell migration in 3D collagen."

{\sl Fibronectin and Related Integrins, Gordon Research Conference} \hfill May 2011 \\
{\bf Meyer, A.S.}, S.K. Hughes-Alford, J.E. Kay, A. Castillo, A. Wells, F.B. Gertler, D.A. Lauffenburger. ``2D protrusion but not motility predicts growth factor-induced cancer cell migration in 3D collagen."

\clearpage

{\sl Fibronectin and Related Integrins, Gordon Research Seminar}, Selected Oral Presentation \hfill May 2011 \\
{\bf Meyer, A.S.}, S.K. Hughes-Alford, A. Wells, F.B. Gertler, D.A. Lauffenburger. ``Heterogeneity of growth factor motility responses among a panel of carcinoma and endometriosis cell lines."

\section{Professional Service}

{\sl Meeting Organizer}, Association of Early Career Cancer Systems Biologists \hfill 2015 -- Present \\
{\sl Ad Hoc Reviewer}, Molecular Cell \hfill 2015 \\
{\sl Member}, Biomedical Engineering Society \hfill 2010 -- Present \\
{\sl Coordinator}, MIT Biological Engineering Graduate Student Board \hfill 2010 -- 2013 \\
{\sl Ad Hoc Reviewer}, Oncogene \hfill 2013 \\
{\sl Ad Hoc Reviewer}, Nature \hfill 2013 \\
{\sl Member}, MIT Biological Engineering Retreat Organizing Committee \hfill 2010 -- 2012 \\
{\sl Ad Hoc Reviewer}, J. Cell Biol. \hfill 2011 - 2012 \\

\section{Patents}

Miller, M.A., M.J. Oudin, A.S. Meyer, L.G. Griffith, F.B. Gertler, D.A. Lauffenburger. ``Diminished proteolytic shedding of receptor tyrosine kinases mediates Mek inhibitor resistance in triple-negative breast cancer." US provisional patent filed.

Meyer, A.S., S.K. Alford, F.B. Gertler, D.A. Lauffenburger. ``Protrusion as a surrogate of 3D motility response." US provisional patent filed.

\end{resume} 
\end{document}