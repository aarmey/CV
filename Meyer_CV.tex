% !TEX TS-program = XeLaTeX
\documentclass[11pt]{res}

\begin{document}

%-------------------------------------
%	YOUR NAME AND ADDRESS(ES) SECTION
%-------------------------------------

\thispagestyle{firststyle}

\name{Aaron S. Meyer\\ \\} % Your name at the top

\address{aameyer@mit.edu \\ (617) 324-4404 \\ \url{http://asmlab.org} } % Your address 1

\address{77 Massachusetts Avenue, 76-361F
 \\ Cambridge, MA 02139} % Your address 2

%---------------------------------------

\begin{resume}

\raggedright 

%	EDUCATION SECTION
\section{Education}

{\sl Ph.D.}, 
Biological Engineering  \hfill April~2014\\ 
Massachusetts Institute of Technology, Cambridge, MA \\ 
Thesis: Quantitative approaches to understanding signaling regulation of 3D cell migration
 
{\sl B.S.}, Bioengineering, magna cum laude \hfill June~2009\\ 
University of California, Los Angeles, CA

%%%%%%%%%%%%%%%%
\section{Research Experience}

{\sl Principal Investigator \& Research Fellow} \hfill September 2014 -- Present \\
Koch Cancer Institute, MIT, Cambridge, MA 
\begin{itemize}
\item Developing systems cancer cell resistance model allowing predictive precision therapy selection
\item Refining models of TAM receptor activation for optimal immunotherapeutic targeting
\end{itemize}

{\sl Postdoctoral Associate in the labs of Forest White \& Douglas Lauffenburger} \hfill June -- September 2014 \\
Department of Biological Engineering \& Koch Cancer Institute, MIT, Cambridge, MA 
\begin{itemize}
\item Performed global phosphotyrosine analysis of TAM receptor transactivation 
\item Utilized systems analysis to identify receptor localization importance to transactivation effects
\end{itemize}
 
{\sl Graduate Researcher in the labs of Douglas Lauffenburger \& Frank Gertler} \hfill 2009 -- 2014 \\
Department of Biological Engineering \& Koch Cancer Institute, MIT, Cambridge, MA 
\begin{itemize}
\item Identified similarities in migration response between dimensionalities, suggesting relevant migration  assays for invasive disease
\item Studied transactivation of TAM receptors and its role in promoting motility response
\item Developed systems models of TAM signaling, unifying conflicting observations regarding the receptors in normal biology and suggesting new methods of intervention to modulate activity
\end{itemize}

{\sl Undergraduate Researcher in the lab of Daniel Kamei} \hfill 2006 -- 2009 \\ 
Department of Bioengineering, University of California, Los Angeles, CA 
\begin{itemize}
\item Investigated biomarker purification using novel aqueous micellar systems
\item Extended a previous statistical mechanics model to nucleic acid partitioning
\item Designed and executed experiments to analyze the partitioning of surfactant systems
\item Developed assays for quantifying the concentration of charged and uncharged surfactants
\end{itemize}

{\sl Summer Intern, Bioprocess Development Division} \hfill 2008 \\ 
Schering-Plough Corporation, Watchung, NJ
\begin{itemize}
\item Developed a novel method for high-throughput batch culture within deep-welled microtiter plates
\item Investigated the social behavior of nonproducing impurities within monoclonal cultures
\item Provided statistical basis for process-based confidence in monoclonality
\end{itemize}





% PUBLICATIONS SECTION
\section{Refereed Publications}
{ \leftskip 0.1in
\parindent -0.1in

Richards, E.J., S. Manole, {\bf A.S. Meyer}. ``Engineering more precise and potent TAM-targeted therapies." {\sl In preparation}.

\parskip 0.1in

Zweemer, A.J.M., C.B. French, J. Mesfin, S. Gordonov, {\bf A.S. Meyer\footnotemark[2]}, D.A. Lauffenburger\footnotemark[2]. ``Apoptotic Cell Bodies Elicit Gas6-Mediated Migration Of AXL-Expressing Tumor Cells." {\sl Submitted.}

Archer, T.C., E.J. Fertig, S.J.C. Gosline, M. Hafner, S.K. Hughes, B.A. Joughin, {\bf A.S. Meyer\footnote{Corresponding author.}}, S.P. Piccolo, A. Shajahan-Haq. ``Systems Approaches to Cancer Biology." {\sl Cancer Research.} 2016. {\sl Accepted}.

Manole, S., E.J. Richards, {\bf A.S. Meyer}. ``JNK pathway activation modulates acquired resistance to EGFR/HER2 targeted therapies." {\sl Cancer Research.} 2016 Sept 15; 76 (18): 5219-5228.

McConnell, R.E., J.E. Van Veen, M. Vidaki, A.V. Kwiatkowski, {\bf A.S. Meyer}, D.A. Lauffenburger, F.B. Gertler. \href{http://jcb.rupress.org/content/213/2/261.full}{``A Requirement for Filopodia Extension Towards Slit During Robo-Mediated Axon Repulsion."} {\sl Journal of Cell Biology.} 2016 Apr 18; 213 (2): 261.

Miller, M.A., M.J. Oudin, R.J. Sullivan, D.T. Frederick, {\bf A.S. Meyer}, S. Wang, H. Im, J. Tadros, L.G. Griffith, H. Lee, R. Weissleder, K.T. Flaherty, F.B. Gertler, D.A. Lauffenburger. \href{http://cancerdiscovery.aacrjournals.org/content/early/2016/03/15/2159-8290.CD-15-0933}{``Reduced proteolytic shedding of receptor tyrosine kinases is a post-translational mechanism of kinase inhibitor resistance."} {\sl Cancer Discovery.}  2016 Apr; 6:331-333.

Miller, M.A., M. Moss, G. Powell, R. Petrovich, L. Edwards, {\bf A.S. Meyer}, L.G. Griffith, D.A. Lauffenburger. \href{http://www.ncbi.nlm.nih.gov/pubmed/26477568}{``Targeting autocrine HB-EGF signaling with specific ADAM12 inhibition using recombinant ADAM12 prodomain."} {\sl Scientific Reports.} 2015 Oct 19; 5:15150.

{\bf Meyer\footnote{Co-corresponding authors.}, A.S.}, A.J.M. Zweemer, D.A. Lauffenburger\footnotemark[\value{footnote}]. \href{http://www.cell.com/cell-systems/abstract/S2405-4712(15)00007-1}{``The AXL receptor is a sensor of ligand spatial heterogeneity."} {\sl Cell Systems.} 2015 Nov 29; 1(1):25-36.

Riquelme, D.N., {\bf A.S. Meyer}, M. Barzik, A. Keating, F.B. Gertler. \href{http://www.ncbi.nlm.nih.gov/pubmed/26221026}{``Selectivity in subunit composition of Ena/VASP tetramers."} {\sl Biosci. Rep.} 2015 Jul 28;35(5). pii: e00246.

{\bf Meyer, A.S.}, M.A. Miller, F.B. Gertler, D.A. Lauffenburger. \href{http://www.ncbi.nlm.nih.gov/pubmed/23921085}{``The receptor AXL diversifies EGFR signaling and limits the response to EGFR-targeted inhibitors in triple-negative breast cancer cells."} {\sl Science Signaling.} 2013 Aug 6; 6(287):ra66.

Miller\footnote{Equally contributing authors.}, M.A., {\bf A.S. Meyer}\footnotemark[\value{footnote}], M. Beste, Z. Lasisi, S. Reddy, K. Jeng, C.-H. Chen, J. Han, K. Isaacson, L.G. Griffith, D.A. Lauffenburger. \href{http://www.ncbi.nlm.nih.gov/pubmed/23674691}{``ADAM-10 and -17 regulate endometriotic cell migration via concerted ligand and receptor shedding feedback on kinase signaling."} {\sl Proc. Natl. Acad. Sci. U.S.A.} 2013 May 28; 110(22):E2074-83.

{\bf Meyer, A.S.}, S.K. Hughes-Alford, J.E. Kay, A. Castillo, A. Wells, F.B. Gertler, D.A. Lauffenburger. \href{http://www.ncbi.nlm.nih.gov/pubmed/22665521}{``2D protrusion but not motility predicts growth factor-induced cancer cell migration in 3D collagen."} {\sl Journal of Cell Biology.} 2012 Jun 11; 197(6):721-9.

Kim, H.D., {\bf A.S. Meyer}, J.P. Wagner, S.K. Alford, A. Wells, F.B. Gertler, D.A. Lauffenburger. \href{http://www.ncbi.nlm.nih.gov/pubmed/21832255}{``Signaling network state predicts Twist-mediated effects on breast cell migration across diverse growth factor contexts."} {\sl Mol. Cell. Proteomics.} 2011 Nov;10(11):M111.008433.

{\bf Meyer, A.S.}, R.G. Condon, G. Keil, N. Jhaveri, Z. Liu, Y.-S. Tsao.  \href{http://www.ncbi.nlm.nih.gov/pubmed/21954223}{``Fluorinert, an oxygen carrier, improves cell culture performance in deep square 96-well plates by facilitating oxygen transfer."} {\sl Biotechnol. Prog.} 2012 Jan; 28(1):171-8.

Mashayekhi, F., {\bf A.S. Meyer}, S.A. Shiigi, V. Nguyen, D.T. Kamei. \href{http://www.ncbi.nlm.nih.gov/pubmed/19061237}{``Concentration of mammalian genomic DNA using two-phase aqueous micellar systems."} {\sl Biotechnol. Bioeng.} 2009 Apr 15; 102(6):1613-23.

}

%--------------------------------------------------------------------

\section{Research Support \& Awards}

{\sl Fellowship Grant} \hfill 2017 -- 2019 \\
Terri Brodeur Breast Cancer Foundation\\
``Decoding the Role of TAM Receptors {\sl In Vivo} Using More Specific and Potent Inhibitors"

\emph{\href{http://www.amgenscholars.com/alumni/ten-to-watch}{Ten to Watch}}, Amgen Scholars Foundation \hfill 2016

{\sl AMIGOS Program Award} \hfill 2016 -- 2020 \\
Jayne Koskinas Ted Giovanis Foundation and Breast Cancer Research Foundation\\
``Understanding the Role of Cell Plasticity in Mediating Drug Resistance"

{\sl Frontier Research Program Initiator Award} \hfill 2015 \\
Koch Institute for Integrative Cancer Research\\
``Multiplexed Tools for Probing Chemokine Receptor Activation State in Breast Cancer"

{\sl NIH Director's Early Independence Award}  \hfill 2014 -- 2019 \\
DP5-OD019815 -- ``Adapter-Layer RTK Signaling: Basic Understanding \& Targeted Drug Resistance" \\
{\bf \href{https://directorsblog.nih.gov/2014/10/28/creative-minds-tackling-chemotherapy-resistance/}{Highlighted by the NIH director's office.}}

{\sl Siebel Scholar, Class of 2014} \hfill 2013 

{\sl Whitaker Fellowship}  \hfill 2013\\
Massachusetts Institute of Technology

{\sl Repligen Fellowship in Cancer Research} \hfill 2012 \\
Koch Institute for Integrative Cancer Research 

{\sl Frontier Research Program Initiator Award} \hfill 2011 \\
Koch Institute for Integrative Cancer Research\\
``Global Growth Factor Reprogramming and Invasion By AXL Expression And Shedding In Breast Carcinoma"

{\sl Breast Cancer Research Predoctoral Fellowship} \hfill 2010 -- 2014 \\
Department of Defense\\
W81XWH-11-1-0088 -- ``Molecular Regulatory Network Dysregulation in Breast Cancer Cell Migration \& Invasion"

{\sl Graduate Research Fellowship}  \hfill 2009 -- 2014 \\
National Science Foundation

{\sl Momenta Presidential Fellowship} \hfill 2009 \\
Massachusetts Institute of Technology



\clearpage
\section{Teaching \& Mentoring Experience}

{\sl Faculty of the Citizen Science Program} \hfill July 2015 -- January 2016 \\
Bard College, Citizen Science Program, Annandale-on-Hudson, NY	
\begin{itemize}
\item Led a short course introducing students to the natural sciences and scientific method
\end{itemize}

{\sl Undergraduate Mentor} \hfill 2009 -- Present \\
MIT, Department of Biological Engineering, Cambridge, MA
\begin{itemize}
\item Designed and supervised projects for nine undergraduate students
\end{itemize}

{\sl Teaching Assistant}, Thermodynamics of Biomolecular Systems \hfill 2010 \\
MIT, Department of Biological Engineering, Cambridge, MA
\begin{itemize}
\item Taught at weekly discussion sections, office hours, and individual appointments
\item Helped write and graded problem sets and exam questions
\end{itemize}

\section{Conference \& Invited Presentations}

{\sl Momenta Pharmaceuticals}, Invited Oral Presentation \hfill April 2017 \\
Robinett, R.A., N. Guan, {\bf A.S. Meyer}. ``Dissecting FcγR Regulation Through a Multivalent Binding Model."

{\sl Univ. of Pennsylvania, Department of Bioengineering}, Invited Departmental Speaker \hfill March 2017 \\
{\bf Meyer, A.S.}. ``Engineering more precise and potent TAM-targeted therapies."

{\sl Univ. of Calif., Los Angeles, Department of Bioengineering}, Invited Departmental Speaker \hfill March 2017 \\
{\bf Meyer, A.S.}. ``Engineering more precise and potent TAM-targeted therapies."

{\sl Biomedical Engineering Society Annual Meeting}, Selected Oral Presentation \hfill October 2016 \\
Manole, S., E.J. Richards, {\bf A.S. Meyer}. ``JNK pathway activation modulates acquired resistance to EGFR/HER2 targeted therapies."

{\sl MD Anderson Cancer Center, Dept. of Systems Biology}, Invited Departmental Speaker \hfill September 2016 \\
Richards, E.J., A. Zweemer, {\bf A.S. Meyer}. ``Engineering more precise and potent TAM-targeted therapies."

{\sl MD Anderson Cancer Center, Future of Science Symposium}, Invited Oral Presentation \hfill September 2016 \\
Manole, S., {\bf A.S. Meyer}. ``Toward precision therapy: Identifying molecular commonalities among RTK bypass resistance mechanisms."

{\sl FASEB Protein Kinase Signaling Network Regulation}, Invited Oral Presentation \hfill July 2016 \\
Richards, E.J., A. Zweemer, {\bf A.S. Meyer}. ``Engineering more precise and potent TAM-targeted therapies."

{\sl Univ. of Calif., Irvine, Center for Complex Biological Systems}, Invited Departmental Speaker \hfill May 2016 \\
Manole, S., E.J. Richards, {\bf A.S. Meyer}. ``Data-driven design of targeted therapies and immunotherapies for cancer."

{\sl Systems Approaches to Cancer Biology}, NCI Invited Oral Presentation \hfill April 2016 \\
Manole, S., E.J. Richards, {\bf A.S. Meyer}. ``Looking across resistance mechanisms to identify molecular commonalities and precision therapy approaches."

{\sl Applied Mathematics in Germinating Oncology Solutions Workshop}, NCI Invited Participant \hfill March 2016

{\sl NIH Common Fund High-Risk High-Reward Symposium} \hfill December 2015 \\
Manole, S., E.J. Richards, {\bf A.S. Meyer}. ``Conserved RTK-intrinsic signaling consequences result in distinct bypass resistance capacity dependent upon pathway dependencies."

{\sl Harvard Medical School, Brugge lab}, Invited Oral Presentation \hfill November 2015 \\
Manole, S., E.J. Richards, {\bf A.S. Meyer}. ``Conserved RTK-intrinsic signaling consequences result in distinct bypass resistance capacity dependent upon pathway dependencies."

{\sl Biomedical Engineering Society Annual Meeting} \hfill October 2015 \\
Manole, S., {\bf A.S. Meyer}. ``Conserved RTK-intrinsic signaling consequences result in distinct bypass resistance capacity dependent upon pathway dependencies."

{\sl ICBP Principal Investigators Meeting} \hfill May 2015 \\
Manole, S., {\bf A.S. Meyer}. ``Conserved RTK-intrinsic signaling consequences result in distinct bypass resistance capacity dependent upon pathway dependencies."

{\sl NIH Common Fund High-Risk High-Reward Symposium} \hfill December 2014 \\
{\bf Meyer, A.S.}. ``Adapter-Layer Integration of RTK Signaling: Basic Understanding and Application to Prediction of Targeted Drug Resistance."

{\sl Biomedical Engineering Society Annual Meeting}, Selected Oral Presentation \hfill October 2014 \\
{\bf Meyer, A.S.}, C.A. Riley, D.A. Lauffenburger. ``AXL Is a Spatial Ligand Differentiation Sensor."

{\sl Interdisciplinary Signaling Workshop}, Selected Oral Presentation \hfill July 2014 \\
{\bf Meyer, A.S.}, C.A. Riley, D.A. Lauffenburger. ``AXL Is a Spatial Ligand Differentiation Sensor."

{\sl ICBP Principal Investigators Meeting} \hfill May 2014 \\
{\bf Meyer, A.S.}, C.A. Riley, D.A. Lauffenburger. ``AXL is a spatial ligand differentiation sensor."

{\sl AACR Molecular Targets and Cancer Therapeutics} \hfill October 2013 \\
{\bf Meyer, A.S.}, F.B. Gertler, D.A. Lauffenburger. ``AXL amplifies EGFR signaling and drives resistance in triple negative breast carcinoma cells."

{\sl Merrimack Pharmaceuticals}, Invited Oral Presentation \hfill October 2013 \\
{\bf Meyer, A.S.}, C.A. Riley, D.A. Lauffenburger. ``AXL is a spatial ligand differentiation sensor."

{\sl ICBP Principal Investigators Meeting} \hfill May 2013 \\
{\bf Meyer, A.S.}, F.B. Gertler, D.A. Lauffenburger. ``AXL amplifies EGFR signaling and drives resistance in triple negative breast carcinoma cells."

{\sl Merrimack Pharmaceuticals}, Invited Oral Presentation \hfill January 2013 \\
{\bf Meyer, A.S.}, F.B. Gertler, D.A. Lauffenburger. ``AXL amplifies EGFR signaling and drives resistance in triple negative breast carcinoma cells."

{\sl PTMs in Cell Signaling, Copenhagen Bioscience Conferences}, Travel Award \hfill December 2012 \\
{\bf Meyer, A.S.}, F.B. Gertler, D.A. Lauffenburger. ``AXL amplifies EGFR signaling and drives resistance in triple negative breast carcinoma cells."

{\sl Biomedical Engineering Society Annual Meeting}, Selected Oral Presentation \hfill October 2012 \\
{\bf Meyer, A.S.}, S.K. Hughes-Alford, J.E. Kay, A. Castillo, A. Wells, F.B. Gertler, D.A. Lauffenburger. ``2D protrusion but not motility predicts growth factor-induced cancer cell migration in 3D collagen."

{\sl Signaling of Adhesion Receptors, Gordon Research Conference} \hfill June 2012 \\
{\bf Meyer, A.S.}, S.K. Hughes-Alford, J.E. Kay, A. Castillo, A. Wells, F.B. Gertler, D.A. Lauffenburger. ``2D protrusion but not motility predicts growth factor-induced cancer cell migration in 3D collagen."

{\sl Systems Biology of Human Disease}, Travel Award \hfill May 2012 \\
{\bf Meyer, A.S.}, S.K. Hughes-Alford, J.E. Kay, A. Castillo, A. Wells, F.B. Gertler, D.A. Lauffenburger. ``2D protrusion but not motility predicts growth factor-induced cancer cell migration in 3D collagen."

{\sl Fibronectin and Related Integrins, Gordon Research Conference} \hfill May 2011 \\
{\bf Meyer, A.S.}, S.K. Hughes-Alford, J.E. Kay, A. Castillo, A. Wells, F.B. Gertler, D.A. Lauffenburger. ``2D protrusion but not motility predicts growth factor-induced cancer cell migration in 3D collagen."

{\sl Fibronectin and Related Integrins, Gordon Research Seminar}, Selected Oral Presentation \hfill May 2011 \\
{\bf Meyer, A.S.}, S.K. Hughes-Alford, A. Wells, F.B. Gertler, D.A. Lauffenburger. ``Heterogeneity of growth factor motility responses among a panel of carcinoma and endometriosis cell lines."



\section{Professional Service}

\newcommand{\serveItem}[3]{{\sl #1}, #2 \hfill #3 \\}

\serveItem{Co-Chair}{Association of Early Career Cancer Systems Biologists}{2017 -- Present}
\serveItem{Graduate Research Fellowship Program Review Panelist}{National Science Foundation}{2016 -- Present}
\serveItem{Meeting Organizer \& Member}{Association of Early Career Cancer Systems Biologists}{2015 -- 2016}
\serveItem{Ad Hoc Reviewer}{Drug Discovery Today}{2016}
\serveItem{Ad Hoc Reviewer}{Molecular Cell}{2015}
\serveItem{Member}{Biomedical Engineering Society}{2010 -- Present}
\serveItem{Coordinator}{MIT Biological Engineering Graduate Student Board}{2010 -- 2013}
\serveItem{Ad Hoc Reviewer}{Oncogene}{2013}
\serveItem{Ad Hoc Reviewer}{Nature}{2013}
\serveItem{Member}{MIT Biological Engineering Retreat Organizing Committee}{2010 -- 2012}
\serveItem{Ad Hoc Reviewer}{J. Cell Biol.}{2011 -- 2012}

\section{Patents/Disclosures}

Richards, E.J., {\bf A.S. Meyer}. ``Receptor Ig domain fragments for specific and potent TAM RTK inhibition." Disclosure filed, 2016.

Richards, E.J., S. Manole, {\bf A.S. Meyer}. ``Modulating JNK activation to impede lung \& breast cancer RTK inhibitor bypass resistance." Disclosure filed, 2016.

Miller, M.A., M.J. Oudin, {\bf A.S. Meyer}, L.G. Griffith, F.B. Gertler, D.A. Lauffenburger. ``Methods of Reducing Kinase Inhibitor Resistance." US patent application 14/690,001,2015.

\end{resume} 
\end{document}
