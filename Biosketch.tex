\documentclass[11pt]{article}

\usepackage{fontspec, microtype, tabto, ragged2e, enumitem, hyperref}
\usepackage[margin=0.5in]{geometry}
\setcounter{secnumdepth}{0}

\setmainfont{Helvetica Neue}

\pagestyle{empty}

\begin{document}
\setlength\parindent{0pt}

\renewcommand{\arraystretch}{1.2}

\begin{flushright}
\footnotesize
	OMB No. 0925-0001 and 0925-0002 (Rev. 10/16 Approved Through 10/31/2018)
\end{flushright}


\begin{tabular}{l|c|c|c}
  \hline
  \multicolumn{4}{c}{\textbf{BIOGRAPHICAL SKETCH}} \\
  \hline
  \multicolumn{4}{l}{NAME: Aaron Samuel Meyer} \\
  \hline
  \multicolumn{4}{l}{eRA COMMONS USER NAME: aameyer1} \\
  \hline
  \multicolumn{4}{l}{POSITION TITLE: Assistant Professor of Bioengineering} \\
  \hline
  \multicolumn{4}{c}{EDUCATION/TRAINING} \\
  \hline
  INSTITUTION AND LOCATION & DEGREE & Completion Date & FIELD OF STUDY \\
  \hline
  University of California, Los Angeles (UCLA) \hspace{40pt} & B.S. & 6/2009 & Bioengineering \\
  Massachusetts Institute of Technology (MIT) & Ph.D. & 6/2014 & Biological Engineering \\
  
\end{tabular}


\vspace{20pt}

\subsection{A. Personal Statement}

% Briefly describe why you are well-suited for your role(s) in the project described in this application. The relevant factors may include aspects of your training; your previous experimental work on this specific topic or related topics; your technical expertise; your collaborators or scientific environment; and your past performance in this or related fields (you may mention specific contributions to science that are not included in Section C).

I am well-suited with the requisite training and expertise to carry out the proposed project herein. I have a background in biological engineering, with training in applied machine learning, receptor-mediated cell signaling, and cancer. My research broadly applies integrated experimental and theoretical approaches to understanding the complex signaling that underlies tumor invasion and immune avoidance. A focus of this work is on the TAMR family of receptor tyrosine kinases, which drive tumor cell resistance, immune avoidance, and efferocytosis of phosphatidylserine-exposing debris. While early in my independent career, I have served as PI on a number of grants and supervised technicians, graduate students, and postdoctoral associates.

This application, while a new undertaking within my research program, builds upon, is strongly influenced by, and is enabled by my earlier work. My experience with both TAMR receptor signaling and data-driven analysis of signaling networks is a critical component to the success of this project. Each Aim of this proposal relies on careful integration of both quantitative experiments and systems-level analysis, a hallmark of my research program.

\begin{enumerate}
  \item \textbf{Meyer, A.S.}$^\dag$, A.J.M. Zweemer, D.A. Lauffenburger$^\dag$. (2015). The AXL Receptor Is a Sensor of Ligand Spatial Heterogeneity. \emph{Cell Systems}, 1(1):25--36. PMCID: PMC4520549.
  \item \textbf{Meyer, A.S.}, M.A. Miller, F.B. Gertler, D.A. Lauffenburger. (2013). The receptor AXL diversifies EGFR signaling and limits the response to EGFR-targeted inhibitors in triple-negative breast cancer cells. \emph{Science Signaling}, 6(287), ra66. PMCID: 3947921.
  \item \textbf{Meyer, A.S.}, S.K. Hughes-Alford, J.E. Kay, A. Castillo, A. Wells, F.B. Gertler, D.A. Lauffenburger (2012). 2D protrusion but not motility predicts growth factor-induced cancer cell migration in 3D collagen. \emph{Journal of Cell Biology}, 197(6), 721-729. PMCID: 3373410.
  \item Miller, M.A.$^\ddag$, \textbf{A.S. Meyer}$^\ddag$, M. Beste, Z. Lasisi, S. Reddy, K. Jeng, C.-H. Chen, J. Han, K. Isaacson, L.G. Griffith, D.A. Lauffenburger. (2013). ADAM-10 and -17 regulate endometriotic cell migration via concerted ligand and receptor shedding feedback on kinase signaling. \emph{Proc. Natl. Acad. Sci. USA}, 110(22), E2074-E2083. PMCID: 3670354.
\end{enumerate}





\subsection{B. Positions and Honors}

\subsubsection{Positions and Employment}

\begin{description}[align=right, labelwidth=2.5cm, font=\normalfont]
\item[2006--2009] Undergraduate Researcher, Bioengineering Department, UCLA
\item[2008] Summer intern, Bioprocess Development Division, Schering-Plough Corporation, Watchung, NJ
\item[2009--2014] Graduate Researcher, Department of Biological Engineering, MIT
\item[2014--2017] Principal Investigator/Research Fellow, Koch Institute for Integrative Cancer Research, MIT
\item[2017--Present] Assistant Professor, Bioengineering Department, UCLA
\end{description}

\subsubsection{Other Experience and Professional Memberships}

\begin{description}[align=right, labelwidth=2.5cm, font=\normalfont]
\item[2010--2012] Member, MIT Biological Engineering Retreat Organizing Committee
\item[2010--2013] Coordinator, MIT Biological Engineering Graduate Student Board
\item[2010--Present] Member, Biomedical Engineering Society
\item[2014--2017] Committee Member, Association of Early Career Cancer Systems Biologists
\item[2015--2016] Organizer, Systems Approaches to Cancer Biology meeting
\item[2017--Present] Chair, Association of Early Career Cancer Systems Biologists
\end{description}

\subsubsection{Honors}

\begin{description}[align=right, labelwidth=2.5cm, font=\normalfont]
\item[2009] Momenta Presidential Fellowship, MIT
\item[2009] Graduate Research Fellowship, National Science Foundation
\item[2010] Breast Cancer Research Predoctoral Fellowship, Department of Defense
\item[2012] Repligen Fellowship in Cancer Research, Koch Institute
\item[2012] Travel grant to attend PTMs in Cell Signaling Conference in Copenhagen, Denmark
\item[2013] Whitaker Fellowship, MIT
\item[2013] Siebel Scholar, Class of 2014
\item[2016] Ten to Watch, Amgen Scholars Foundation
\item[2016--2017] Fellowship, Terri Brodeur Breast Cancer Foundation
\end{description}





\subsection{C. Contribution to Science}

% Briefly describe up to five of your most significant contributions to science. For each contribution, indicate the historical background that frames the scientific problem; the central finding(s); the influence of the finding(s) on the progress of science or the application of those finding(s) to health or technology; and your specific role in the described work. For each of these contributions, reference up to four peer-reviewed publications or other non-publication research products (can include audio or video products; patents; data and research materials; databases; educational aids or curricula; instruments or equipment; models; protocols; and software or netware) that are relevant to the described contribution. The description of each contribution should be no longer than one half page including figures and citations. Also provide a URL to a full list of your published work as found in a publicly available digital database such as SciENcv or My Bibliography, which are maintained by the US National Library of Medicine.

\subsubsection{TAMR receptor signaling}

TAMR receptors play diverse roles in cancer progression as elucidated through genetic manipulation but their mechanisms of post-translational regulation are complex. As efferocytosis receptors, TAMRs have dual roles in mediating transport of extracellular debris as well as intracellular signaling. While small molecule and biologic TAMR inhibitors are in preclinical and clinical development, knowing where and in which patients TAMR therapies will be effective will aid translation of these therapies. These studies focused on the factors that drive AXL activation within carcinoma cells, and in turn the consequence of this activation. We first identified that AXL can be transactivated from ErbB receptors and that this transactivation drives the invasiveness of breast carcinoma cells more so than the signaling from the ErbB receptors themselves (2). Then, we developed a kinetic model of AXL activation, mechanistically explaining the dependence of the receptor upon phosphatidylserine for activation (1). This basic understanding of AXL signaling will allow for more rationally designed therapies and understanding of which factors in the tumor microenvironment lead to activation.

\begin{enumerate}
  \item \textbf{Meyer, A.S.}$^\dag$, A.J.M. Zweemer, D.A. Lauffenburger$^\dag$. (2015). The AXL Receptor Is a Sensor of Ligand Spatial Heterogeneity. \emph{Cell Systems}, 1(1):25--36. PMCID: 4520549.\\ Associated software repository: \url{https://github.com/meyer-lab/AXLdiffEQ}
  \item \textbf{Meyer, A.S.}, M.A. Miller, F.B. Gertler, D.A. Lauffenburger. (2013). The receptor AXL diversifies EGFR signaling and limits the response to EGFR-targeted inhibitors in triple-negative breast cancer cells. \emph{Science Signaling}, 6(287), ra66. PMCID: 3947921.
\end{enumerate}

\noindent $^\dag$Co-corresponding authors.






\subsubsection{Therapeutic resistance and design} % OK

The benefits cancer patients derive from targeted therapies are limited by genetic and non-genetic mechanisms of resistance. This is in part due to an incomplete understanding of the many compensatory molecular changes that occur when one treats with a therapy. In (1) we explored a panel of resistance mechanisms to RTK inhibitors, showed that coordinate JNK/Erk/Akt measurement was essential to predict cellular outcome, and showed that the resistance mechanism's effects could be explained through their effects on these pathways. In (3) we showed that a complication of targeting autocrine growth factor signaling is the length-scales on which ligand release and recapture occur. Through a diffusion reaction model, we instead predicted and showed that inhibiting ligand release through protease inhibition is much more effective. In (2), we showed that a common consequence of kinase inhibitors is reduced proteolytic shedding on the cell surface. This change switches the kinase dependence of cells, in turn driving resistance to therapy (in large part via AXL). These results highlight the complexity underlying targeted inhibitor response and demonstrate methods to understand and overcome it.

\begin{enumerate}
  \item Manole, S., E.J. Richards, {\bf A.S. Meyer}. JNK pathway activation modulates acquired resistance to EGFR/HER2 targeted therapies. {\sl Cancer Research.} 2016 Sept 15; 76 (18): 5219-5228. PMCID: 5026573.\\ Associated software repository: \url{https://github.com/thanatosmin/resistance-modeling}
  \item Miller, M.A., M.J. Oudin, R.J. Sullivan, S.J. Wang, \textbf{A.S. Meyer}, H. Im, D.T. Frederick, J. Tadros, L.G. Griffith, H. Lee, R. Weissleder, K.T. Flaherty, F.B. Gertler, D.A. Lauffenburger. (2016). Reduced Proteolytic Shedding of Receptor Tyrosine Kinases is a Post-Translational Mechanism of Kinase Inhibitor Resistance. \emph{Cancer Discovery}, 6(4):331--333, April 2016. PMCID: 5087317.
  \item M.A. Miller, M.L. Moss, G. Powell, R. Petrovich, L. Edwards, \textbf{A.S. Meyer}, Linda G. Griffith, D.A. Lauffenburger. Targeting autocrine HB-EGF signaling with specific ADAM12 inhibition using recombinant ADAM12 prodomain. \emph{Scientific Reports}, 5:15150 EP --, October 2015. PMCID: 4609913.
\end{enumerate}





\subsubsection{Migration and metastasis mechanisms}

Invasion and dissemination of cells underlies many diseases including breast cancer. Studying these processes is complicated by their regulation on multiple levels, and the multiple biophysical steps involved. In earlier work, we studied cell migration overall and individual processes involved in cell migration, then compared them to 3D invasion through extracellular matrix (1). This identified that individual processes still regulated migration in 3D, but that the overall rate-limiting steps and thus migration response were different. By studying the signaling (3) and protease (2) regulation of migration, we then linked these processes to the invasive response to identify therapeutic approaches.

\begin{enumerate}
  \item \textbf{Meyer, A.S.}, S.K. Hughes-Alford, J.E. Kay, A. Castillo, A. Wells, F.B. Gertler, D.A. Lauffenburger (2012). 2D protrusion but not motility predicts growth factor-induced cancer cell migration in 3D collagen. \emph{Journal of Cell Biology}, 197(6), 721-729. PMCID: 3373410.
  \item Miller, M.A.$^\ddag$, \textbf{A.S. Meyer}$^\ddag$, M. Beste, Z. Lasisi, S. Reddy, Jeng, K., Chen, C.-H., Han, J., Isaacson, K., Griffith, L.G., Lauffenburger, D.A. (2013). ADAM-10 and -17 regulate endometriotic cell migration via concerted ligand and receptor shedding feedback on kinase signaling. \emph{Proc. Natl. Acad. Sci. USA}, 110(22), E2074-E2083. PMCID: 3670354.
  \item Kim, H.D., \textbf{Meyer, A.S.}, Wagner, J.P., Alford, S.K., Wells, A., Gertler, F.B., Lauffenburger, D.A. (2011). Signaling network state predicts Twist-mediated effects on breast cell migration across diverse growth factor contexts. \emph{Molecular \& Cellular Proteomics}, 10(11), M111.008433. PMCID: 3226401.
  \item Riquelme, D.N., \textbf{A.S. Meyer}, M. Barzik, A. Keating, F.B. Gertler. (2015). Selectivity in Subunit Composition of Ena/VASP Tetramers. \emph{Bioscience Reports}, 2015. PMCID: 4721544.
\end{enumerate}

\noindent $^\ddag$Equal contribution.




\textbf{Complete List of Published Work in My Bibliography: \url{http://1.usa.gov/1So8BFr}}




\NumTabs{3}

\subsection{D. Research Support}

\subsubsection{Ongoing Research Support}

AMIGOS Program Award  \hfill  12/1/2016--1/1/2020

Jayne Koskinas Ted Giovanis Foundation \& Breast Cancer Research Foundation

Understanding the Role of Cell Plasticity in Mediating Drug Resistance

This project aims to study how cell state plasticity contributes to drug resistance and use an integrated experimental and modeling approach to identify rational approaches to overcome plasticity-induced resistance.

Role: Co-PI

\vspace{12pt}

NIH DP5-OD019815  \hfill  9/22/2014--9/1/2019

Adapter-Layer RTK Signaling: Basic Understanding \& Targeted Drug Resistance

The goal of this project is to study sets of resistance mechanisms to RTK-targeted therapies, in order to identify commonalities and ways to determine which mechanism may be driving individual tumors. There is no overlap with this proposal.

Role: PI

\subsubsection{Completed Research Support}

Frontier Research Program Initiator Award, Koch Institute, MIT \hfill 6/1/2015--6/1/2016

Quantitative and Multiplexed Tools for Probing G-Protein Coupled Receptor Activation

The goal of this project was to develop a novel multiplexed G-protein activation assay able to globally assess their activity.

Role: Co-PI

\vspace{12pt}



W81XWH-11-1-0088  \hfill  12/1/2010--12/1/2013

Multivariate analysis of 3D breast cancer cell invasion

The goal of this project was a systems analysis of breast cancer cell migration and invasion, connecting 3D migration to signaling changes during EMT.

Role: PI

\end{document}
